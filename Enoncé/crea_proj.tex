Dans cette partie, nous nous intéresserons à la création de projectiles que pourra lancer le véhicule. Ces projectiles seront ce que l'on appelle des \textit{prefabs}, c'est à dire des objets instantiables (un peu comme des classes en POO).

\begin{enumerate}
\item Ajoutez un objet dans la scène (vous pouvez choisir une sphère, ou un modèle de votre choix). Sa position n'a pas d'importance. Vous pouvez la personnaliser à votre convenance (vous pouvez par exemple lui donner une lumière de sorte à le faire briller). Notez qu'il lui faudra un composant de type \textit{RigidBody} pour pouvoir être déplacé par la suite.
\item Transformez votre objet en prefab en le faisant glisser dans le dossier \textit{Assets/Prefabs}, puis supprimer celui présent dans la scène.
\item Complétez le script \textit{ProjectileSpawner.cs} et attachez-le à un objet de la scène pour permettre au véhicule de faire apparaître les projectiles lors de l'appui d'une touche et leur donne une vitesse initiale. Le squelette de ce script est donné dans \textit{Assets/Scripts/ProjectileSpawner.cs}. Vous pourrez éventuellement créer un élément vide fils de \textit{Player} servant de point de départ des projectiles.
\end{enumerate}