\documentclass[a4paper]{article}

\usepackage[english]{babel}
\usepackage[utf8]{inputenc}
\usepackage{amsmath}
\usepackage{graphicx}
\usepackage[colorinlistoftodos]{todonotes}

\title{TP Unity}

\author{Alexandre Audinot Thomas Giraudeau \\
Gabriel Prevosto Dan Seeruttun--Marie}

%\date{\today}

\begin{document}
\maketitle

\section{Mise en route} \input{"mise_en_route.tex"}
\section{Création du décor} \input{"crea_decor.tex"}
\section{Création du joueur} \input{"crea_joueur.tex"}
\section{Création des projectiles} \input{"crea_proj.tex"}
\section{Partie libre}
Si vous êtes arrivés jusque là, bravo ! Vous pouvez toujours vous donner des objectifs supplémentaires (ajout de cibles, d'un score, rendre les arbres solides...).
\section{Liens utiles}
\begin{itemize}
\item Documentation Unity : http://docs.unity3d.com/Manual/index.html
\item Documentation de l'API C\# : http://docs.unity3d.com/ScriptReference/
\item Un point de départ pour les déplacements : http://docs.unity3d.com/ScriptReference/Rigidbody.html
\end{itemize}

\end{document}